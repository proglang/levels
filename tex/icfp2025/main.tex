\documentclass[manuscript,screen,review,anonymous]{acmart}
%%
%% \BibTeX command to typeset BibTeX logo in the docs
\AtBeginDocument{%
  \providecommand\BibTeX{{%
    Bib\TeX}}}

%% Rights management information.  This information is sent to you
%% when you complete the rights form.  These commands have SAMPLE
%% values in them; it is your responsibility as an author to replace
%% the commands and values with those provided to you when you
%% complete the rights form.
\setcopyright{acmlicensed}
\copyrightyear{2018}
\acmYear{2018}
\acmDOI{XXXXXXX.XXXXXXX}
%% These commands are for a PROCEEDINGS abstract or paper.
\acmConference[Conference acronym 'XX]{Make sure to enter the correct
  conference title from your rights confirmation email}{June 03--05,
  2018}{Woodstock, NY}
%%
%%  Uncomment \acmBooktitle if the title of the proceedings is different
%%  from ``Proceedings of ...''!
%%
%%\acmBooktitle{Woodstock '18: ACM Symposium on Neural Gaze Detection,
%%  June 03--05, 2018, Woodstock, NY}
\acmISBN{978-1-4503-XXXX-X/2018/06}


%%
%% For managing citations, it is recommended to use bibliography
%% files in BibTeX format.
%%
%% You can then either use BibTeX with the ACM-Reference-Format style,
%% or BibLaTeX with the acmnumeric or acmauthoryear sytles, that include
%% support for advanced citation of software artefact from the
%% biblatex-software package, also separately available on CTAN.
%%
%% Look at the sample-*-biblatex.tex files for templates showcasing
%% the biblatex styles.
%%

%%
%% The majority of ACM publications use numbered citations and
%% references.  The command \citestyle{authoryear} switches to the
%% "author year" style.
%%
%% If you are preparing content for an event
%% sponsored by ACM SIGGRAPH, you must use the "author year" style of
%% citations and references.
%% Uncommenting
%% the next command will enable that style.
%%\citestyle{acmauthoryear}


%%
%% end of the preamble, start of the body of the document source.
\begin{document}

%%
%% The "title" command has an optional parameter,
%% allowing the author to define a "short title" to be used in page headers.
\title{There is Life in the Universes Beyond $\omega$}

%%
%% The "author" command and its associated commands are used to define
%% the authors and their affiliations.
%% Of note is the shared affiliation of the first two authors, and the
%% "authornote" and "authornotemark" commands
%% used to denote shared contribution to the research.
\author{Marius Weidner}
\email{weidner@cs.uni-freiburgde}
\orcid{0009-0008-1152-165X}

\author{Peter Thiemann}
\email{thiemann@acm.org}
\orcid{0000-0002-9000-1239}

\author{Hannes Saffrich}
\email{saffrich@cs.uni-freiburg.de}
\affiliation{%
  \institution{University of Freiburg}
  % \city{Hekla}
  \country{Germany}}
\orcid{0009-0004-7014-754X}


%%
%% The abstract is a short summary of the work to be presented in the
%% article.
\begin{abstract}
  The first draft of Martin-L\"{o}f's type theory proposed the
  assumption Type:Type. Universe levels have been introduced to
  avoid the resulting inconsistencies by assuming
  Type$_i$:Type$_{i+1}$. 
  Proof assistants based on type theory support such universe levels
  to varying degree, but they impose restrictions that can make coding
  awkward.

  Specifically, we consider the ramifications of Agda's
  approach to handling levels using a denotational semantics of a
  stratified version of System F as a motivating example.
  We propose a simple fix that extends Agda's capabilities for handling
  universe levels parametrically up to $\varepsilon_0$.
\end{abstract}

%%
%% The code below is generated by the tool at http://dl.acm.org/ccs.cfm.
%% Please copy and paste the code instead of the example below.
%%
% \begin{CCSXML}
% \end{CCSXML}

% \ccsdesc[500]{Do Not Use This Code~Generate the Correct Terms for Your Paper}
% \ccsdesc[300]{Do Not Use This Code~Generate the Correct Terms for Your Paper}
% \ccsdesc{Do Not Use This Code~Generate the Correct Terms for Your Paper}
% \ccsdesc[100]{Do Not Use This Code~Generate the Correct Terms for Your Paper}

%%
%% Keywords. The author(s) should pick words that accurately describe
%% the work being presented. Separate the keywords with commas.
\keywords{Dependent types, universes, ordinal numbers}


%%
%% This command processes the author and affiliation and title
%% information and builds the first part of the formatted document.
\maketitle

\section{Introduction}
\label{sec:introduction}

The origin of universe levels.

How do universe levels work in Agda?

What becomes awkward with Agda's approach?

How do we propose to fix it?

Contributions.


\section{Preliminaries}
\label{sec:preliminaries}

Agda, Ordinals, IR-Universes

\section{Constructions}
\label{sec:constructions}




\section{Related Work}
\label{sec:related-work}

How do other proof assistants (Coq, Lean) handle universes?
Cumulativity, Impact of impredicativity


\section{Conclusions}
\label{sec:conclusions}


%%
%% The acknowledgments section is defined using the "acks" environment
%% (and NOT an unnumbered section). This ensures the proper
%% identification of the section in the article metadata, and the
%% consistent spelling of the heading.
\begin{acks}
  To whom it may concern.
\end{acks}

%%
%% The next two lines define the bibliography style to be used, and
%% the bibliography file.
\bibliographystyle{ACM-Reference-Format}
\bibliography{../references}


\end{document}
\endinput
